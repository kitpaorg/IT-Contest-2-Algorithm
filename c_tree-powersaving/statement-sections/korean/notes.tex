예제에서 밥이 따르는 절차와 그에 따른 전구의 상태는 아래와 같다.

\begin{enumerate}
  \item 밥이 $u=3$번 정점으로 이동하고, $v=4$번 정점의 전구가 꺼진다.
  \item 밥이 $u=2$번 정점으로 이동하고, $v=3$번 정점의 전구가 꺼진다.
  \item 밥이 $u=1$번 정점으로 이동하고, $v=2$번 정점의 전구가 켜진다.
  \item 밥이 $u=2$번 정점으로 이동하고, $v=1$번 정점의 전구가 꺼진다.
  \item 밥이 $u=3$번 정점으로 이동하고, $v=2$번 정점의 전구가 꺼진다.
  \item 밥이 $u=5$번 정점으로 이동하고, $v=3$번 정점의 전구가 켜진다.
  \item 밥이 $u=3$번 정점으로 이동하고, $v=5$번 정점의 전구가 켜진다.
  \item 밥이 $u=5$번 정점으로 이동하고, $v=3$번 정점의 전구가 꺼진다.
  \item 밥이 $u=3$번 정점으로 이동하고, $v=5$번 정점의 전구가 꺼진다.
  \item 밥이 $u=6$번 정점으로 이동하고, $v=3$번 정점의 전구가 켜진다.
  \item 밥이 $u=3$번 정점으로 이동하고, $v=6$번 정점의 전구가 꺼진다.
  \item 밥이 $u=4$번 정점으로 이동하고, $v=3$번 정점의 전구가 꺼진다.
\end{enumerate}

총 $12$번의 절차를 시행하였으며, 절차가 끝난 후 트리의 전구는 모두 꺼진 상태이다.
